% Options for packages loaded elsewhere
\PassOptionsToPackage{unicode}{hyperref}
\PassOptionsToPackage{hyphens}{url}
%
\documentclass[
]{article}
\usepackage{amsmath,amssymb}
\usepackage{lmodern}
\usepackage{iftex}
\ifPDFTeX
  \usepackage[T1]{fontenc}
  \usepackage[utf8]{inputenc}
  \usepackage{textcomp} % provide euro and other symbols
\else % if luatex or xetex
  \usepackage{unicode-math}
  \defaultfontfeatures{Scale=MatchLowercase}
  \defaultfontfeatures[\rmfamily]{Ligatures=TeX,Scale=1}
\fi
% Use upquote if available, for straight quotes in verbatim environments
\IfFileExists{upquote.sty}{\usepackage{upquote}}{}
\IfFileExists{microtype.sty}{% use microtype if available
  \usepackage[]{microtype}
  \UseMicrotypeSet[protrusion]{basicmath} % disable protrusion for tt fonts
}{}
\makeatletter
\@ifundefined{KOMAClassName}{% if non-KOMA class
  \IfFileExists{parskip.sty}{%
    \usepackage{parskip}
  }{% else
    \setlength{\parindent}{0pt}
    \setlength{\parskip}{6pt plus 2pt minus 1pt}}
}{% if KOMA class
  \KOMAoptions{parskip=half}}
\makeatother
\usepackage{xcolor}
\usepackage[margin=1in]{geometry}
\usepackage{color}
\usepackage{fancyvrb}
\newcommand{\VerbBar}{|}
\newcommand{\VERB}{\Verb[commandchars=\\\{\}]}
\DefineVerbatimEnvironment{Highlighting}{Verbatim}{commandchars=\\\{\}}
% Add ',fontsize=\small' for more characters per line
\usepackage{framed}
\definecolor{shadecolor}{RGB}{248,248,248}
\newenvironment{Shaded}{\begin{snugshade}}{\end{snugshade}}
\newcommand{\AlertTok}[1]{\textcolor[rgb]{0.94,0.16,0.16}{#1}}
\newcommand{\AnnotationTok}[1]{\textcolor[rgb]{0.56,0.35,0.01}{\textbf{\textit{#1}}}}
\newcommand{\AttributeTok}[1]{\textcolor[rgb]{0.77,0.63,0.00}{#1}}
\newcommand{\BaseNTok}[1]{\textcolor[rgb]{0.00,0.00,0.81}{#1}}
\newcommand{\BuiltInTok}[1]{#1}
\newcommand{\CharTok}[1]{\textcolor[rgb]{0.31,0.60,0.02}{#1}}
\newcommand{\CommentTok}[1]{\textcolor[rgb]{0.56,0.35,0.01}{\textit{#1}}}
\newcommand{\CommentVarTok}[1]{\textcolor[rgb]{0.56,0.35,0.01}{\textbf{\textit{#1}}}}
\newcommand{\ConstantTok}[1]{\textcolor[rgb]{0.00,0.00,0.00}{#1}}
\newcommand{\ControlFlowTok}[1]{\textcolor[rgb]{0.13,0.29,0.53}{\textbf{#1}}}
\newcommand{\DataTypeTok}[1]{\textcolor[rgb]{0.13,0.29,0.53}{#1}}
\newcommand{\DecValTok}[1]{\textcolor[rgb]{0.00,0.00,0.81}{#1}}
\newcommand{\DocumentationTok}[1]{\textcolor[rgb]{0.56,0.35,0.01}{\textbf{\textit{#1}}}}
\newcommand{\ErrorTok}[1]{\textcolor[rgb]{0.64,0.00,0.00}{\textbf{#1}}}
\newcommand{\ExtensionTok}[1]{#1}
\newcommand{\FloatTok}[1]{\textcolor[rgb]{0.00,0.00,0.81}{#1}}
\newcommand{\FunctionTok}[1]{\textcolor[rgb]{0.00,0.00,0.00}{#1}}
\newcommand{\ImportTok}[1]{#1}
\newcommand{\InformationTok}[1]{\textcolor[rgb]{0.56,0.35,0.01}{\textbf{\textit{#1}}}}
\newcommand{\KeywordTok}[1]{\textcolor[rgb]{0.13,0.29,0.53}{\textbf{#1}}}
\newcommand{\NormalTok}[1]{#1}
\newcommand{\OperatorTok}[1]{\textcolor[rgb]{0.81,0.36,0.00}{\textbf{#1}}}
\newcommand{\OtherTok}[1]{\textcolor[rgb]{0.56,0.35,0.01}{#1}}
\newcommand{\PreprocessorTok}[1]{\textcolor[rgb]{0.56,0.35,0.01}{\textit{#1}}}
\newcommand{\RegionMarkerTok}[1]{#1}
\newcommand{\SpecialCharTok}[1]{\textcolor[rgb]{0.00,0.00,0.00}{#1}}
\newcommand{\SpecialStringTok}[1]{\textcolor[rgb]{0.31,0.60,0.02}{#1}}
\newcommand{\StringTok}[1]{\textcolor[rgb]{0.31,0.60,0.02}{#1}}
\newcommand{\VariableTok}[1]{\textcolor[rgb]{0.00,0.00,0.00}{#1}}
\newcommand{\VerbatimStringTok}[1]{\textcolor[rgb]{0.31,0.60,0.02}{#1}}
\newcommand{\WarningTok}[1]{\textcolor[rgb]{0.56,0.35,0.01}{\textbf{\textit{#1}}}}
\usepackage{graphicx}
\makeatletter
\def\maxwidth{\ifdim\Gin@nat@width>\linewidth\linewidth\else\Gin@nat@width\fi}
\def\maxheight{\ifdim\Gin@nat@height>\textheight\textheight\else\Gin@nat@height\fi}
\makeatother
% Scale images if necessary, so that they will not overflow the page
% margins by default, and it is still possible to overwrite the defaults
% using explicit options in \includegraphics[width, height, ...]{}
\setkeys{Gin}{width=\maxwidth,height=\maxheight,keepaspectratio}
% Set default figure placement to htbp
\makeatletter
\def\fps@figure{htbp}
\makeatother
\setlength{\emergencystretch}{3em} % prevent overfull lines
\providecommand{\tightlist}{%
  \setlength{\itemsep}{0pt}\setlength{\parskip}{0pt}}
\setcounter{secnumdepth}{-\maxdimen} % remove section numbering
\ifLuaTeX
  \usepackage{selnolig}  % disable illegal ligatures
\fi
\IfFileExists{bookmark.sty}{\usepackage{bookmark}}{\usepackage{hyperref}}
\IfFileExists{xurl.sty}{\usepackage{xurl}}{} % add URL line breaks if available
\urlstyle{same} % disable monospaced font for URLs
\hypersetup{
  pdftitle={R\_studio},
  pdfauthor={Christoffer Mondrup Kramer},
  hidelinks,
  pdfcreator={LaTeX via pandoc}}

\title{R\_studio}
\author{Christoffer Mondrup Kramer}
\date{2023-04-22}

\begin{document}
\maketitle

\hypertarget{ex.-2a-09-02-2023}{%
\section{Ex. 2a: 09-02-2023}\label{ex.-2a-09-02-2023}}

\begin{Shaded}
\begin{Highlighting}[]
\FunctionTok{library}\NormalTok{(tidyverse)}
\end{Highlighting}
\end{Shaded}

\begin{verbatim}
## -- Attaching core tidyverse packages ------------------------ tidyverse 2.0.0 --
## v dplyr     1.1.2     v readr     2.1.4
## v forcats   1.0.0     v stringr   1.5.0
## v ggplot2   3.4.2     v tibble    3.2.1
## v lubridate 1.9.2     v tidyr     1.3.0
## v purrr     1.0.1     
## -- Conflicts ------------------------------------------ tidyverse_conflicts() --
## x dplyr::filter() masks stats::filter()
## x dplyr::lag()    masks stats::lag()
## i Use the conflicted package (<http://conflicted.r-lib.org/>) to force all conflicts to become errors
\end{verbatim}

\begin{Shaded}
\begin{Highlighting}[]
\FunctionTok{library}\NormalTok{(}\StringTok{"scatterplot3d"}\NormalTok{)}
\end{Highlighting}
\end{Shaded}

\hypertarget{custom-functions}{%
\subsection{Custom functions}\label{custom-functions}}

Here is the collections for custom functions I have created myself

\begin{Shaded}
\begin{Highlighting}[]
\CommentTok{\#Plotting marginal dot diagram}
\NormalTok{margin\_dot\_plot }\OtherTok{\textless{}{-}} \ControlFlowTok{function}\NormalTok{(x, y, }\AttributeTok{xlabel =} \StringTok{"x"}\NormalTok{, }\AttributeTok{ylabel =} \StringTok{"y"}\NormalTok{) \{}
\FunctionTok{layout}\NormalTok{(}\AttributeTok{mat =} \FunctionTok{matrix}\NormalTok{(}\FunctionTok{c}\NormalTok{(}\DecValTok{2}\NormalTok{, }\DecValTok{0}\NormalTok{, }\CommentTok{\# First column}
                      \DecValTok{1}\NormalTok{, }\DecValTok{3}\NormalTok{), }\CommentTok{\# Second column }
                        \AttributeTok{nrow =} \DecValTok{2}\NormalTok{, }
                        \AttributeTok{ncol =} \DecValTok{2}\NormalTok{),}
       \AttributeTok{heights =} \FunctionTok{c}\NormalTok{(}\DecValTok{6}\NormalTok{, }\DecValTok{2}\NormalTok{),    }\CommentTok{\# Heights of the two rows}
       \AttributeTok{widths =} \FunctionTok{c}\NormalTok{(}\DecValTok{1}\NormalTok{, }\DecValTok{6}\NormalTok{))     }\CommentTok{\# Widths of the two columns}
\FunctionTok{par}\NormalTok{(}\AttributeTok{mar =} \FunctionTok{c}\NormalTok{(}\DecValTok{4}\NormalTok{, }\CommentTok{\# Bottom}
            \DecValTok{4}\NormalTok{, }\CommentTok{\# left}
            \FloatTok{0.1}\NormalTok{, }\CommentTok{\# Top}
            \FloatTok{0.1}\NormalTok{))}\CommentTok{\# Right}

\FunctionTok{plot}\NormalTok{(x, y, }\AttributeTok{xlab =} \StringTok{""}\NormalTok{, }\AttributeTok{ylab =} \StringTok{""}\NormalTok{)}

\FunctionTok{stripchart}\NormalTok{(y, }\AttributeTok{method =} \StringTok{"stack"}\NormalTok{, }\AttributeTok{at =} \DecValTok{0}\NormalTok{,}
           \AttributeTok{pch =} \DecValTok{16}\NormalTok{, }\AttributeTok{col =} \StringTok{"darkgreen"}\NormalTok{, }\AttributeTok{frame =} \ConstantTok{FALSE}\NormalTok{, }\AttributeTok{vertical =} \ConstantTok{TRUE}\NormalTok{, }\AttributeTok{ylab =}\NormalTok{ ylabel)}
\FunctionTok{stripchart}\NormalTok{(x, }\AttributeTok{method =} \StringTok{"stack"}\NormalTok{, }\AttributeTok{at =} \DecValTok{0}\NormalTok{,}
           \AttributeTok{pch =} \DecValTok{16}\NormalTok{, }\AttributeTok{col =} \StringTok{"darkgreen"}\NormalTok{, }\AttributeTok{frame =} \ConstantTok{FALSE}\NormalTok{, }\AttributeTok{xlab =}\NormalTok{ xlabel)}
\NormalTok{\}}


\CommentTok{\# Arithmetic mean (pp. 6{-}7)}
\NormalTok{my\_mean }\OtherTok{\textless{}{-}} \ControlFlowTok{function}\NormalTok{(my\_list) \{}
\NormalTok{  n }\OtherTok{=} \FunctionTok{length}\NormalTok{(my\_list)}
  \FunctionTok{return}\NormalTok{( (}\DecValTok{1}\SpecialCharTok{/}\NormalTok{n) }\SpecialCharTok{*} \FunctionTok{sum}\NormalTok{(my\_list))}
\NormalTok{\}}


\CommentTok{\# Variance for single variable (p .7)}
\NormalTok{my\_single\_sample\_variance }\OtherTok{\textless{}{-}} \ControlFlowTok{function}\NormalTok{(my\_list) \{}
\NormalTok{  n }\OtherTok{=} \FunctionTok{length}\NormalTok{((my\_list))}
\NormalTok{  x\_mean }\OtherTok{=} \FunctionTok{mean}\NormalTok{(my\_list)}
  
\NormalTok{  total\_res }\OtherTok{=} \DecValTok{0}
  \ControlFlowTok{for}\NormalTok{ (x }\ControlFlowTok{in}\NormalTok{ my\_list) \{}
\NormalTok{    inner\_res }\OtherTok{=}\NormalTok{ (x }\SpecialCharTok{{-}}\NormalTok{ x\_mean)}\SpecialCharTok{\^{}}\DecValTok{2}
\NormalTok{    total\_res }\OtherTok{=}\NormalTok{ inner\_res }\SpecialCharTok{+}\NormalTok{ total\_res}
\NormalTok{  \}}
  \FunctionTok{return}\NormalTok{ ((}\DecValTok{1}\SpecialCharTok{/}\NormalTok{(n }\SpecialCharTok{{-}} \DecValTok{1}\NormalTok{)) }\SpecialCharTok{*}\NormalTok{ total\_res)}
\NormalTok{\}}

\CommentTok{\# Co variance (p. 7)}
\NormalTok{my\_sample\_covar }\OtherTok{\textless{}{-}} \ControlFlowTok{function}\NormalTok{(my\_list\_1, my\_list\_2) \{}
\NormalTok{  n }\OtherTok{\textless{}{-}} \FunctionTok{length}\NormalTok{(my\_list\_1)}

\NormalTok{  list\_1\_mean }\OtherTok{\textless{}{-}} \FunctionTok{my\_mean}\NormalTok{(my\_list\_1)}
\NormalTok{  list\_2\_mean }\OtherTok{\textless{}{-}} \FunctionTok{my\_mean}\NormalTok{(my\_list\_2)}
  
\NormalTok{  res }\OtherTok{\textless{}{-}} \DecValTok{0}
  \ControlFlowTok{for}\NormalTok{ (i }\ControlFlowTok{in} \DecValTok{1}\SpecialCharTok{:}\NormalTok{n) \{}
\NormalTok{    list\_1\_res }\OtherTok{\textless{}{-}}\NormalTok{ my\_list\_1[i] }\SpecialCharTok{{-}}\NormalTok{ list\_1\_mean}
\NormalTok{    list\_2\_res }\OtherTok{\textless{}{-}}\NormalTok{ my\_list\_2[i] }\SpecialCharTok{{-}}\NormalTok{ list\_2\_mean}
\NormalTok{    temp\_res }\OtherTok{\textless{}{-}}\NormalTok{ (list\_1\_res }\SpecialCharTok{*}\NormalTok{ list\_2\_res)}
\NormalTok{    res }\OtherTok{\textless{}{-}}\NormalTok{ res }\SpecialCharTok{+}\NormalTok{ temp\_res}
\NormalTok{  \}}
  \FunctionTok{return}\NormalTok{((}\DecValTok{1}\SpecialCharTok{/}\NormalTok{(n}\DecValTok{{-}1}\NormalTok{)) }\SpecialCharTok{*}\NormalTok{ res)}
\NormalTok{\}}

\CommentTok{\# Correlation coefficient (p. 8)}
\NormalTok{my\_cor\_coef }\OtherTok{\textless{}{-}} \ControlFlowTok{function}\NormalTok{(my\_list\_1, my\_list\_2)\{}
\NormalTok{  covariance }\OtherTok{\textless{}{-}} \FunctionTok{my\_sample\_covar}\NormalTok{(my\_list\_1, my\_list\_2)}
\NormalTok{  var\_list\_1 }\OtherTok{\textless{}{-}} \FunctionTok{my\_single\_sample\_variance}\NormalTok{(my\_list\_1)}
\NormalTok{  var\_list\_2 }\OtherTok{\textless{}{-}} \FunctionTok{my\_single\_sample\_variance}\NormalTok{(my\_list\_2)}
  
\NormalTok{  res }\OtherTok{\textless{}{-}}\NormalTok{ covariance }\SpecialCharTok{/}\NormalTok{ ( }\FunctionTok{sqrt}\NormalTok{(var\_list\_1) }\SpecialCharTok{*} \FunctionTok{sqrt}\NormalTok{(var\_list\_2) )}
  \FunctionTok{return}\NormalTok{(res)}
\NormalTok{\}}


\CommentTok{\# Mean array}
\NormalTok{my\_mean\_array }\OtherTok{\textless{}{-}} \ControlFlowTok{function}\NormalTok{(df)\{}
\NormalTok{  mean\_array }\OtherTok{\textless{}{-}} \FunctionTok{numeric}\NormalTok{()}
  
\NormalTok{  i }\OtherTok{\textless{}{-}} \DecValTok{1}
  \ControlFlowTok{for}\NormalTok{ (colname }\ControlFlowTok{in} \FunctionTok{colnames}\NormalTok{(df)) \{}
\NormalTok{    mean\_array[i] }\OtherTok{\textless{}{-}} \FunctionTok{my\_mean}\NormalTok{(df[,colname])}
\NormalTok{    i }\OtherTok{\textless{}{-}}\NormalTok{ i }\SpecialCharTok{+} \DecValTok{1}
\NormalTok{  \}}
  \FunctionTok{return}\NormalTok{(mean\_array)}
\NormalTok{\}}
\end{Highlighting}
\end{Shaded}

\hypertarget{p.-38}{%
\subsection{1.4 - p.~38}\label{p.-38}}

The world's 10 largest companies yields the following data:

\begin{Shaded}
\begin{Highlighting}[]
\NormalTok{company }\OtherTok{\textless{}{-}} \FunctionTok{c}\NormalTok{(}\StringTok{"Citigroup"}\NormalTok{, }\StringTok{"General Electric"}\NormalTok{, }\StringTok{"American Intl Group"}\NormalTok{, }\StringTok{"Bank of America"}\NormalTok{, }\StringTok{"HSBC Group"}\NormalTok{, }\StringTok{"ExonMobil"}\NormalTok{, }\StringTok{"Royal Dutch/shell"}\NormalTok{, }\StringTok{"BP"}\NormalTok{, }\StringTok{"ING Group"}\NormalTok{, }\StringTok{"Toyota Motor"}\NormalTok{)}

\CommentTok{\# x1}
\NormalTok{sales }\OtherTok{\textless{}{-}} \FunctionTok{c}\NormalTok{(}\FloatTok{108.28}\NormalTok{, }\FloatTok{152.36}\NormalTok{, }\FloatTok{95.04}\NormalTok{, }\FloatTok{65.45}\NormalTok{, }\FloatTok{62.97}\NormalTok{, }\FloatTok{263.99}\NormalTok{, }\FloatTok{265.19}\NormalTok{, }\FloatTok{285.06}\NormalTok{, }\FloatTok{92.01}\NormalTok{, }\FloatTok{165.68}\NormalTok{)}

\CommentTok{\# x2}
\NormalTok{profits }\OtherTok{\textless{}{-}} \FunctionTok{c}\NormalTok{(}\FloatTok{17.05}\NormalTok{, }\FloatTok{16.59}\NormalTok{, }\FloatTok{10.91}\NormalTok{, }\FloatTok{14.14}\NormalTok{, }\FloatTok{9.52}\NormalTok{, }\FloatTok{25.33}\NormalTok{, }\FloatTok{18.54}\NormalTok{, }\FloatTok{15.73}\NormalTok{, }\FloatTok{8.10}\NormalTok{, }\FloatTok{11.13}\NormalTok{)}

\CommentTok{\# x3}
\NormalTok{assets }\OtherTok{\textless{}{-}} \FunctionTok{c}\NormalTok{(}\FloatTok{1484.10}\NormalTok{, }\FloatTok{750.33}\NormalTok{, }\FloatTok{766.42}\NormalTok{, }\FloatTok{1110.46}\NormalTok{, }\FloatTok{1031.29}\NormalTok{, }\FloatTok{195.26}\NormalTok{, }\FloatTok{193.83}\NormalTok{, }\FloatTok{191.11}\NormalTok{, }\FloatTok{1175.16}\NormalTok{, }\FloatTok{211.15}\NormalTok{)}

\NormalTok{company\_df }\OtherTok{\textless{}{-}} \FunctionTok{data.frame}\NormalTok{(company,}
\NormalTok{                         sales,}
\NormalTok{                         profits,}
\NormalTok{                         assets)}
\end{Highlighting}
\end{Shaded}

\begin{Shaded}
\begin{Highlighting}[]
\FunctionTok{print}\NormalTok{(company\_df)}
\end{Highlighting}
\end{Shaded}

\begin{verbatim}
##                company  sales profits  assets
## 1            Citigroup 108.28   17.05 1484.10
## 2     General Electric 152.36   16.59  750.33
## 3  American Intl Group  95.04   10.91  766.42
## 4      Bank of America  65.45   14.14 1110.46
## 5           HSBC Group  62.97    9.52 1031.29
## 6            ExonMobil 263.99   25.33  195.26
## 7    Royal Dutch/shell 265.19   18.54  193.83
## 8                   BP 285.06   15.73  191.11
## 9            ING Group  92.01    8.10 1175.16
## 10        Toyota Motor 165.68   11.13  211.15
\end{verbatim}

\hypertarget{a-plot-the-scatter-diagram-and-the-marignal-dot-diagrams-for-variables-x_1-and-x_2}{%
\subsubsection{\texorpdfstring{a) Plot the scatter diagram and the
marignal dot diagrams for variables \(x_1\) and
\(x_2\)}{a) Plot the scatter diagram and the marignal dot diagrams for variables x\_1 and x\_2}}\label{a-plot-the-scatter-diagram-and-the-marignal-dot-diagrams-for-variables-x_1-and-x_2}}

\begin{Shaded}
\begin{Highlighting}[]
\FunctionTok{margin\_dot\_plot}\NormalTok{(company\_df}\SpecialCharTok{$}\NormalTok{sales, company\_df}\SpecialCharTok{$}\NormalTok{profits, }\AttributeTok{xlabel =}  \StringTok{"Sales"}\NormalTok{, }\AttributeTok{ylabel =} \StringTok{"Profits"}\NormalTok{)}
\end{Highlighting}
\end{Shaded}

\includegraphics{ex_2a_files/figure-latex/unnamed-chunk-5-1.pdf}

\hypertarget{b-compute-barx_1-barx_2-s_11-s_22-s_12-r_12.-interpret-r_12}{%
\subsubsection{\texorpdfstring{b) Compute \(\bar{x}_1\), \(\bar{x}_2\),
\(s_{11}\), \(s_{22}\), \(s_{12}\), \(r_{12}\). Interpret
\(r_{12}\)}{b) Compute \textbackslash bar\{x\}\_1, \textbackslash bar\{x\}\_2, s\_\{11\}, s\_\{22\}, s\_\{12\}, r\_\{12\}. Interpret r\_\{12\}}}\label{b-compute-barx_1-barx_2-s_11-s_22-s_12-r_12.-interpret-r_12}}

\begin{Shaded}
\begin{Highlighting}[]
\CommentTok{\#x1 mean}
\FunctionTok{print}\NormalTok{(}\StringTok{"x1 mean"}\NormalTok{)}
\end{Highlighting}
\end{Shaded}

\begin{verbatim}
## [1] "x1 mean"
\end{verbatim}

\begin{Shaded}
\begin{Highlighting}[]
\FunctionTok{my\_mean}\NormalTok{(company\_df}\SpecialCharTok{$}\NormalTok{sales)}
\end{Highlighting}
\end{Shaded}

\begin{verbatim}
## [1] 155.603
\end{verbatim}

\begin{Shaded}
\begin{Highlighting}[]
\FunctionTok{mean}\NormalTok{(company\_df}\SpecialCharTok{$}\NormalTok{sales)}
\end{Highlighting}
\end{Shaded}

\begin{verbatim}
## [1] 155.603
\end{verbatim}

\begin{Shaded}
\begin{Highlighting}[]
\CommentTok{\# x2 mean}
\FunctionTok{print}\NormalTok{(}\StringTok{"x2 mean"}\NormalTok{)}
\end{Highlighting}
\end{Shaded}

\begin{verbatim}
## [1] "x2 mean"
\end{verbatim}

\begin{Shaded}
\begin{Highlighting}[]
\FunctionTok{my\_mean}\NormalTok{(company\_df}\SpecialCharTok{$}\NormalTok{profits)}
\end{Highlighting}
\end{Shaded}

\begin{verbatim}
## [1] 14.704
\end{verbatim}

\begin{Shaded}
\begin{Highlighting}[]
\FunctionTok{mean}\NormalTok{(company\_df}\SpecialCharTok{$}\NormalTok{profits)}
\end{Highlighting}
\end{Shaded}

\begin{verbatim}
## [1] 14.704
\end{verbatim}

\begin{Shaded}
\begin{Highlighting}[]
\CommentTok{\#s11 (variance)}
\FunctionTok{print}\NormalTok{(}\StringTok{"x1 variance"}\NormalTok{)}
\end{Highlighting}
\end{Shaded}

\begin{verbatim}
## [1] "x1 variance"
\end{verbatim}

\begin{Shaded}
\begin{Highlighting}[]
\FunctionTok{my\_single\_sample\_variance}\NormalTok{(company\_df}\SpecialCharTok{$}\NormalTok{sales)}
\end{Highlighting}
\end{Shaded}

\begin{verbatim}
## [1] 7476.453
\end{verbatim}

\begin{Shaded}
\begin{Highlighting}[]
\FunctionTok{var}\NormalTok{(company\_df}\SpecialCharTok{$}\NormalTok{sales)}
\end{Highlighting}
\end{Shaded}

\begin{verbatim}
## [1] 7476.453
\end{verbatim}

\begin{Shaded}
\begin{Highlighting}[]
\CommentTok{\# s22 (variance)}
\FunctionTok{print}\NormalTok{(}\StringTok{"x2 variance"}\NormalTok{)}
\end{Highlighting}
\end{Shaded}

\begin{verbatim}
## [1] "x2 variance"
\end{verbatim}

\begin{Shaded}
\begin{Highlighting}[]
\FunctionTok{my\_single\_sample\_variance}\NormalTok{(company\_df}\SpecialCharTok{$}\NormalTok{profits)}
\end{Highlighting}
\end{Shaded}

\begin{verbatim}
## [1] 26.19032
\end{verbatim}

\begin{Shaded}
\begin{Highlighting}[]
\FunctionTok{var}\NormalTok{(company\_df}\SpecialCharTok{$}\NormalTok{profits)}
\end{Highlighting}
\end{Shaded}

\begin{verbatim}
## [1] 26.19032
\end{verbatim}

\begin{Shaded}
\begin{Highlighting}[]
\CommentTok{\# s12 (covariance)}
\FunctionTok{print}\NormalTok{(}\StringTok{"x1 and x2 covariance"}\NormalTok{)}
\end{Highlighting}
\end{Shaded}

\begin{verbatim}
## [1] "x1 and x2 covariance"
\end{verbatim}

\begin{Shaded}
\begin{Highlighting}[]
\FunctionTok{my\_sample\_covar}\NormalTok{(company\_df}\SpecialCharTok{$}\NormalTok{sales, company\_df}\SpecialCharTok{$}\NormalTok{profits)}
\end{Highlighting}
\end{Shaded}

\begin{verbatim}
## [1] 303.6186
\end{verbatim}

\begin{Shaded}
\begin{Highlighting}[]
\FunctionTok{cov}\NormalTok{(company\_df}\SpecialCharTok{$}\NormalTok{sales, company\_df}\SpecialCharTok{$}\NormalTok{profits)}
\end{Highlighting}
\end{Shaded}

\begin{verbatim}
## [1] 303.6186
\end{verbatim}

\begin{Shaded}
\begin{Highlighting}[]
\CommentTok{\# r12 (correlation coefficient)}
\FunctionTok{print}\NormalTok{(}\StringTok{"Correlation coeficient"}\NormalTok{)}
\end{Highlighting}
\end{Shaded}

\begin{verbatim}
## [1] "Correlation coeficient"
\end{verbatim}

\begin{Shaded}
\begin{Highlighting}[]
\FunctionTok{my\_cor\_coef}\NormalTok{(company\_df}\SpecialCharTok{$}\NormalTok{sales, company\_df}\SpecialCharTok{$}\NormalTok{profits)}
\end{Highlighting}
\end{Shaded}

\begin{verbatim}
## [1] 0.686136
\end{verbatim}

\begin{Shaded}
\begin{Highlighting}[]
\FunctionTok{cor}\NormalTok{(company\_df}\SpecialCharTok{$}\NormalTok{sales, company\_df}\SpecialCharTok{$}\NormalTok{profits)}
\end{Highlighting}
\end{Shaded}

\begin{verbatim}
## [1] 0.686136
\end{verbatim}

Since the r value or correlation is above 0, there is a positive
correlation between sales and profits. Ie. the more sales the higher
profits. Which also makes perfect sense. Since it is more than 0.5 there
is even a strong positive correlation. see
\url{https://www.scribbr.com/statistics/pearson-correlation-coefficient}

\hypertarget{use-the-data-from-previously}{%
\subsection{1.5 Use the data from
previously}\label{use-the-data-from-previously}}

\hypertarget{a-plot-the-scatter-and-dot-diagrams-for-x_2-x_3-and-x_1-x_3.-comment-on-the-patterns}{%
\subsubsection{\texorpdfstring{a) Plot the scatter and dot diagrams for
\((x_2, x_3)\) and \((x_1, x_3)\). Comment on the
patterns}{a) Plot the scatter and dot diagrams for (x\_2, x\_3) and (x\_1, x\_3). Comment on the patterns}}\label{a-plot-the-scatter-and-dot-diagrams-for-x_2-x_3-and-x_1-x_3.-comment-on-the-patterns}}

\begin{Shaded}
\begin{Highlighting}[]
\NormalTok{x\_1 }\OtherTok{\textless{}{-}}\NormalTok{ company\_df}\SpecialCharTok{$}\NormalTok{sales}
\NormalTok{x\_2 }\OtherTok{\textless{}{-}}\NormalTok{ company\_df}\SpecialCharTok{$}\NormalTok{profits}
\NormalTok{x\_3 }\OtherTok{\textless{}{-}}\NormalTok{ company\_df}\SpecialCharTok{$}\NormalTok{assets}

\NormalTok{new\_df }\OtherTok{\textless{}{-}} \FunctionTok{data.frame}\NormalTok{(x\_1,}
\NormalTok{                     x\_2,}
\NormalTok{                     x\_3)}
\FunctionTok{print}\NormalTok{(new\_df)}
\end{Highlighting}
\end{Shaded}

\begin{verbatim}
##       x_1   x_2     x_3
## 1  108.28 17.05 1484.10
## 2  152.36 16.59  750.33
## 3   95.04 10.91  766.42
## 4   65.45 14.14 1110.46
## 5   62.97  9.52 1031.29
## 6  263.99 25.33  195.26
## 7  265.19 18.54  193.83
## 8  285.06 15.73  191.11
## 9   92.01  8.10 1175.16
## 10 165.68 11.13  211.15
\end{verbatim}

\((x_2, x_3)\)

\begin{Shaded}
\begin{Highlighting}[]
\FunctionTok{margin\_dot\_plot}\NormalTok{(x\_2, x\_3, }\AttributeTok{xlabel =} \StringTok{"Profits (x\_2)"}\NormalTok{, }\AttributeTok{ylabel =} \StringTok{"Assets (x\_3)"}\NormalTok{)}
\end{Highlighting}
\end{Shaded}

\includegraphics{ex_2a_files/figure-latex/unnamed-chunk-8-1.pdf} In
General it is hard to see a clear pattern between the profits and the
assets visually. IOf anything it might be negative correlated, but I am
unsure. The varuance seem to be higher for the profits and assets seem
to clump more together. \((x_1, x_3)\)

\begin{Shaded}
\begin{Highlighting}[]
\FunctionTok{margin\_dot\_plot}\NormalTok{(x\_1, x\_3, }\AttributeTok{xlabel =} \StringTok{"Sales (x\_1)"}\NormalTok{, }\AttributeTok{ylabel =} \StringTok{"Assets (x\_3"}\NormalTok{)}
\end{Highlighting}
\end{Shaded}

\includegraphics{ex_2a_files/figure-latex/unnamed-chunk-9-1.pdf} here
there seem to be a negative correlation, with more sales leading to
fewer assets, which might indicate that they are emptying there stock,
this might explain the negative correlation before. Sales are likewiser
more variance, but have a tendency to cluster.

\hypertarget{compute-the-barx-s_n-r}{%
\subsubsection{\texorpdfstring{Compute the
\textbf{\(\bar{x}, S_n, R\)}}{Compute the \textbackslash bar\{x\}, S\_n, R}}\label{compute-the-barx-s_n-r}}

\$\bar\{x\}

\begin{Shaded}
\begin{Highlighting}[]
\FunctionTok{my\_mean\_array}\NormalTok{(new\_df)}
\end{Highlighting}
\end{Shaded}

\begin{verbatim}
## [1] 155.603  14.704 710.911
\end{verbatim}

\begin{Shaded}
\begin{Highlighting}[]
\FunctionTok{colMeans}\NormalTok{(new\_df)}
\end{Highlighting}
\end{Shaded}

\begin{verbatim}
##     x_1     x_2     x_3 
## 155.603  14.704 710.911
\end{verbatim}

\(S_n\)

\begin{Shaded}
\begin{Highlighting}[]
\FunctionTok{cov}\NormalTok{(new\_df)}
\end{Highlighting}
\end{Shaded}

\begin{verbatim}
##             x_1         x_2        x_3
## x_1   7476.4532   303.61862 -35575.960
## x_2    303.6186    26.19032  -1053.827
## x_3 -35575.9596 -1053.82739 237054.270
\end{verbatim}

\(R\)

\begin{Shaded}
\begin{Highlighting}[]
\FunctionTok{cor}\NormalTok{(new\_df}
\NormalTok{    )}
\end{Highlighting}
\end{Shaded}

\begin{verbatim}
##            x_1        x_2        x_3
## x_1  1.0000000  0.6861360 -0.8450549
## x_2  0.6861360  1.0000000 -0.4229366
## x_3 -0.8450549 -0.4229366  1.0000000
\end{verbatim}

\hypertarget{p.-40}{%
\subsection{1.7 (p.~40)}\label{p.-40}}

You are given the following n = 3 observations and p = 2 variables:

\begin{Shaded}
\begin{Highlighting}[]
\NormalTok{x\_1 }\OtherTok{\textless{}{-}} \FunctionTok{c}\NormalTok{(}\DecValTok{2}\NormalTok{, }\DecValTok{3}\NormalTok{, }\DecValTok{4}\NormalTok{)}
\NormalTok{x\_2 }\OtherTok{\textless{}{-}} \FunctionTok{c}\NormalTok{(}\DecValTok{1}\NormalTok{, }\DecValTok{2}\NormalTok{, }\DecValTok{4}\NormalTok{)}

\NormalTok{df }\OtherTok{\textless{}{-}} \FunctionTok{data.frame}\NormalTok{(x\_1,}
\NormalTok{                 x\_2)}
\FunctionTok{print}\NormalTok{(df)}
\end{Highlighting}
\end{Shaded}

\begin{verbatim}
##   x_1 x_2
## 1   2   1
## 2   3   2
## 3   4   4
\end{verbatim}

\hypertarget{a-plot-the-pairs-of-observations-in-the-two-dimensional-variable-space.-that-is-construct-a-two-dimensional-scatter-plot-of-the-data}{%
\subsubsection{a) Plot the pairs of observations in the two dimensional
variable space. That is, construct a two-dimensional scatter plot of the
data}\label{a-plot-the-pairs-of-observations-in-the-two-dimensional-variable-space.-that-is-construct-a-two-dimensional-scatter-plot-of-the-data}}

\begin{Shaded}
\begin{Highlighting}[]
\FunctionTok{pairs}\NormalTok{(df)}
\end{Highlighting}
\end{Shaded}

\includegraphics{ex_2a_files/figure-latex/unnamed-chunk-14-1.pdf}

\begin{Shaded}
\begin{Highlighting}[]
\FunctionTok{plot}\NormalTok{(x\_1, x\_2)}
\end{Highlighting}
\end{Shaded}

\includegraphics{ex_2a_files/figure-latex/unnamed-chunk-15-1.pdf}

\begin{Shaded}
\begin{Highlighting}[]
\FunctionTok{plot}\NormalTok{(x\_2, x\_1)}
\end{Highlighting}
\end{Shaded}

\includegraphics{ex_2a_files/figure-latex/unnamed-chunk-16-1.pdf}

\hypertarget{b-plot-the-data-as-two-points-in-the-three-dimensional-item-space}{%
\subsubsection{b) Plot the data as two points in the three dimensional
item
space}\label{b-plot-the-data-as-two-points-in-the-three-dimensional-item-space}}

\begin{Shaded}
\begin{Highlighting}[]
\NormalTok{transposed }\OtherTok{\textless{}{-}} \FunctionTok{t}\NormalTok{(df)}
\FunctionTok{print}\NormalTok{(transposed)}
\end{Highlighting}
\end{Shaded}

\begin{verbatim}
##     [,1] [,2] [,3]
## x_1    2    3    4
## x_2    1    2    4
\end{verbatim}

\begin{Shaded}
\begin{Highlighting}[]
\FunctionTok{scatterplot3d}\NormalTok{(}\FunctionTok{t}\NormalTok{(df))}
\end{Highlighting}
\end{Shaded}

\includegraphics{ex_2a_files/figure-latex/unnamed-chunk-17-1.pdf} A bit
prettier

\begin{Shaded}
\begin{Highlighting}[]
\FunctionTok{scatterplot3d}\NormalTok{(}\AttributeTok{x =} \FunctionTok{c}\NormalTok{(}\DecValTok{2}\NormalTok{,}\DecValTok{1}\NormalTok{), }\AttributeTok{y =} \FunctionTok{c}\NormalTok{(}\DecValTok{3}\NormalTok{,}\DecValTok{2}\NormalTok{), }\AttributeTok{z =} \FunctionTok{c}\NormalTok{(}\DecValTok{4}\NormalTok{,}\DecValTok{4}\NormalTok{))}
\end{Highlighting}
\end{Shaded}

\includegraphics{ex_2a_files/figure-latex/unnamed-chunk-18-1.pdf} \#\#
1.27 (p.~46) Table 1.11 presents the 2005 attendence (millions) at the
fifteen most visisted national parks and their size (acres)

\begin{Shaded}
\begin{Highlighting}[]
\NormalTok{parks\_df }\OtherTok{\textless{}{-}} \FunctionTok{read.table}\NormalTok{(}\StringTok{"T1{-}11.dat"}\NormalTok{)}
\FunctionTok{colnames}\NormalTok{(parks\_df)[}\DecValTok{1}\NormalTok{] }\OtherTok{\textless{}{-}} \StringTok{"size"}
\FunctionTok{colnames}\NormalTok{(parks\_df)[}\DecValTok{2}\NormalTok{] }\OtherTok{\textless{}{-}} \StringTok{"visitors"}
\FunctionTok{print}\NormalTok{(parks\_df)}
\end{Highlighting}
\end{Shaded}

\begin{verbatim}
##      size visitors
## 1    47.4     2.05
## 2    35.8     1.02
## 3    32.9     2.53
## 4  1508.5     1.23
## 5  1217.4     4.40
## 6   310.0     2.46
## 7   521.8     9.19
## 8     5.6     1.34
## 9   922.7     3.14
## 10  235.6     1.17
## 11  265.8     2.80
## 12  199.0     1.09
## 13 2219.8     2.84
## 14  761.3     3.30
## 15  146.6     2.59
\end{verbatim}

\hypertarget{a-create-a-scatter-plot-and-calculate-the-correlation-coeeficient}{%
\subsubsection{a) Create a scatter plot and calculate the correlation
coeeficient}\label{a-create-a-scatter-plot-and-calculate-the-correlation-coeeficient}}

\begin{Shaded}
\begin{Highlighting}[]
\FunctionTok{plot}\NormalTok{(parks\_df)}
\end{Highlighting}
\end{Shaded}

\includegraphics{ex_2a_files/figure-latex/unnamed-chunk-20-1.pdf}

\begin{Shaded}
\begin{Highlighting}[]
\FunctionTok{cor}\NormalTok{(parks\_df)}
\end{Highlighting}
\end{Shaded}

\begin{verbatim}
##               size  visitors
## size     1.0000000 0.1725274
## visitors 0.1725274 1.0000000
\end{verbatim}

\hypertarget{b-identify-the-park-that-is-unusual.-drop-this-point-and-recalculate-the-correlation-coefficient.-comment-on-the-effect-of-this-point-correlation.}{%
\subsubsection{B) Identify the park that is unusual. Drop this point and
recalculate the correlation coefficient. Comment on the effect of this
point
correlation.}\label{b-identify-the-park-that-is-unusual.-drop-this-point-and-recalculate-the-correlation-coefficient.-comment-on-the-effect-of-this-point-correlation.}}

We do this by adding names to the plot to detect the outlier

\begin{Shaded}
\begin{Highlighting}[]
\FunctionTok{plot}\NormalTok{(parks\_df)}
\FunctionTok{text}\NormalTok{(parks\_df,}
     \AttributeTok{labels=}\FunctionTok{rownames}\NormalTok{(parks\_df),}
     \AttributeTok{cex=} \FloatTok{0.8}\NormalTok{, }\CommentTok{\# Font size}
     \AttributeTok{pos=}\DecValTok{2}\NormalTok{) }\CommentTok{\# Position}
\end{Highlighting}
\end{Shaded}

\includegraphics{ex_2a_files/figure-latex/unnamed-chunk-22-1.pdf} It
appears to be value 7 or thirteen, I will say that is is number 7 since
it has a ridiculousness number of visitors

\begin{Shaded}
\begin{Highlighting}[]
\FunctionTok{print}\NormalTok{(parks\_df[}\FunctionTok{c}\NormalTok{(}\DecValTok{7}\NormalTok{, }\DecValTok{13}\NormalTok{), ])}
\end{Highlighting}
\end{Shaded}

\begin{verbatim}
##      size visitors
## 7   521.8     9.19
## 13 2219.8     2.84
\end{verbatim}

Both are unsual, I will try removing one at a time and then both

No number 7

\begin{Shaded}
\begin{Highlighting}[]
\FunctionTok{print}\NormalTok{(}\StringTok{"original"}\NormalTok{)}
\end{Highlighting}
\end{Shaded}

\begin{verbatim}
## [1] "original"
\end{verbatim}

\begin{Shaded}
\begin{Highlighting}[]
\FunctionTok{cor}\NormalTok{(parks\_df)}
\end{Highlighting}
\end{Shaded}

\begin{verbatim}
##               size  visitors
## size     1.0000000 0.1725274
## visitors 0.1725274 1.0000000
\end{verbatim}

\begin{Shaded}
\begin{Highlighting}[]
\FunctionTok{print}\NormalTok{(}\StringTok{" "}\NormalTok{)}
\end{Highlighting}
\end{Shaded}

\begin{verbatim}
## [1] " "
\end{verbatim}

\begin{Shaded}
\begin{Highlighting}[]
\FunctionTok{print}\NormalTok{(}\StringTok{" "}\NormalTok{)}
\end{Highlighting}
\end{Shaded}

\begin{verbatim}
## [1] " "
\end{verbatim}

\begin{Shaded}
\begin{Highlighting}[]
\FunctionTok{print}\NormalTok{(}\StringTok{" "}\NormalTok{)}
\end{Highlighting}
\end{Shaded}

\begin{verbatim}
## [1] " "
\end{verbatim}

\begin{Shaded}
\begin{Highlighting}[]
\FunctionTok{print}\NormalTok{(}\StringTok{"No number 7"}\NormalTok{)}
\end{Highlighting}
\end{Shaded}

\begin{verbatim}
## [1] "No number 7"
\end{verbatim}

\begin{Shaded}
\begin{Highlighting}[]
\FunctionTok{cor}\NormalTok{(parks\_df[}\SpecialCharTok{{-}}\DecValTok{7}\NormalTok{, ])}
\end{Highlighting}
\end{Shaded}

\begin{verbatim}
##               size  visitors
## size     1.0000000 0.3907829
## visitors 0.3907829 1.0000000
\end{verbatim}

\begin{Shaded}
\begin{Highlighting}[]
\FunctionTok{print}\NormalTok{(}\StringTok{" "}\NormalTok{)}
\end{Highlighting}
\end{Shaded}

\begin{verbatim}
## [1] " "
\end{verbatim}

\begin{Shaded}
\begin{Highlighting}[]
\FunctionTok{print}\NormalTok{(}\StringTok{" "}\NormalTok{)}
\end{Highlighting}
\end{Shaded}

\begin{verbatim}
## [1] " "
\end{verbatim}

\begin{Shaded}
\begin{Highlighting}[]
\FunctionTok{print}\NormalTok{(}\StringTok{" "}\NormalTok{)}
\end{Highlighting}
\end{Shaded}

\begin{verbatim}
## [1] " "
\end{verbatim}

\begin{Shaded}
\begin{Highlighting}[]
\FunctionTok{print}\NormalTok{(}\StringTok{"No number 13"}\NormalTok{)}
\end{Highlighting}
\end{Shaded}

\begin{verbatim}
## [1] "No number 13"
\end{verbatim}

\begin{Shaded}
\begin{Highlighting}[]
\FunctionTok{cor}\NormalTok{(parks\_df[}\SpecialCharTok{{-}}\DecValTok{13}\NormalTok{, ])}
\end{Highlighting}
\end{Shaded}

\begin{verbatim}
##               size  visitors
## size     1.0000000 0.2299564
## visitors 0.2299564 1.0000000
\end{verbatim}

\begin{Shaded}
\begin{Highlighting}[]
\FunctionTok{print}\NormalTok{(}\StringTok{" "}\NormalTok{)}
\end{Highlighting}
\end{Shaded}

\begin{verbatim}
## [1] " "
\end{verbatim}

\begin{Shaded}
\begin{Highlighting}[]
\FunctionTok{print}\NormalTok{(}\StringTok{" "}\NormalTok{)}
\end{Highlighting}
\end{Shaded}

\begin{verbatim}
## [1] " "
\end{verbatim}

\begin{Shaded}
\begin{Highlighting}[]
\FunctionTok{print}\NormalTok{(}\StringTok{" "}\NormalTok{)}
\end{Highlighting}
\end{Shaded}

\begin{verbatim}
## [1] " "
\end{verbatim}

\begin{Shaded}
\begin{Highlighting}[]
\FunctionTok{print}\NormalTok{(}\StringTok{"No 13 or 7"}\NormalTok{)}
\end{Highlighting}
\end{Shaded}

\begin{verbatim}
## [1] "No 13 or 7"
\end{verbatim}

\begin{Shaded}
\begin{Highlighting}[]
\FunctionTok{cor}\NormalTok{(parks\_df[}\FunctionTok{c}\NormalTok{(}\SpecialCharTok{{-}}\DecValTok{7}\NormalTok{, }\SpecialCharTok{{-}}\DecValTok{13}\NormalTok{), ])}
\end{Highlighting}
\end{Shaded}

\begin{verbatim}
##              size visitors
## size     1.000000 0.398539
## visitors 0.398539 1.000000
\end{verbatim}

There is suddenly a moderat positive correlation between size and
visitors, and it appears that number thirteen is not really the big
outlier, since it has very little effect on the correlation.

\hypertarget{c-would-the-correlation-in-part-b-change-if-you-measure-in-size-in-square-miles-instead-of-acres-explain.}{%
\subsubsection{c) Would the correlation in part b change if you measure
in size in square miles instead of acres?
Explain.}\label{c-would-the-correlation-in-part-b-change-if-you-measure-in-size-in-square-miles-instead-of-acres-explain.}}

Let us test it:

\begin{Shaded}
\begin{Highlighting}[]
\NormalTok{parks\_df}\SpecialCharTok{$}\NormalTok{size }\OtherTok{\textless{}{-}}\NormalTok{ parks\_df}\SpecialCharTok{$}\NormalTok{size }\SpecialCharTok{/} \DecValTok{640}

\FunctionTok{print}\NormalTok{(parks\_df)}
\end{Highlighting}
\end{Shaded}

\begin{verbatim}
##          size visitors
## 1  0.07406250     2.05
## 2  0.05593750     1.02
## 3  0.05140625     2.53
## 4  2.35703125     1.23
## 5  1.90218750     4.40
## 6  0.48437500     2.46
## 7  0.81531250     9.19
## 8  0.00875000     1.34
## 9  1.44171875     3.14
## 10 0.36812500     1.17
## 11 0.41531250     2.80
## 12 0.31093750     1.09
## 13 3.46843750     2.84
## 14 1.18953125     3.30
## 15 0.22906250     2.59
\end{verbatim}

\begin{Shaded}
\begin{Highlighting}[]
\FunctionTok{print}\NormalTok{(}\StringTok{"original"}\NormalTok{)}
\end{Highlighting}
\end{Shaded}

\begin{verbatim}
## [1] "original"
\end{verbatim}

\begin{Shaded}
\begin{Highlighting}[]
\FunctionTok{cor}\NormalTok{(parks\_df)}
\end{Highlighting}
\end{Shaded}

\begin{verbatim}
##               size  visitors
## size     1.0000000 0.1725274
## visitors 0.1725274 1.0000000
\end{verbatim}

\begin{Shaded}
\begin{Highlighting}[]
\FunctionTok{print}\NormalTok{(}\StringTok{" "}\NormalTok{)}
\end{Highlighting}
\end{Shaded}

\begin{verbatim}
## [1] " "
\end{verbatim}

\begin{Shaded}
\begin{Highlighting}[]
\FunctionTok{print}\NormalTok{(}\StringTok{" "}\NormalTok{)}
\end{Highlighting}
\end{Shaded}

\begin{verbatim}
## [1] " "
\end{verbatim}

\begin{Shaded}
\begin{Highlighting}[]
\FunctionTok{print}\NormalTok{(}\StringTok{" "}\NormalTok{)}
\end{Highlighting}
\end{Shaded}

\begin{verbatim}
## [1] " "
\end{verbatim}

\begin{Shaded}
\begin{Highlighting}[]
\FunctionTok{print}\NormalTok{(}\StringTok{"No number 7"}\NormalTok{)}
\end{Highlighting}
\end{Shaded}

\begin{verbatim}
## [1] "No number 7"
\end{verbatim}

\begin{Shaded}
\begin{Highlighting}[]
\FunctionTok{cor}\NormalTok{(parks\_df[}\SpecialCharTok{{-}}\DecValTok{7}\NormalTok{, ])}
\end{Highlighting}
\end{Shaded}

\begin{verbatim}
##               size  visitors
## size     1.0000000 0.3907829
## visitors 0.3907829 1.0000000
\end{verbatim}

\begin{Shaded}
\begin{Highlighting}[]
\FunctionTok{print}\NormalTok{(}\StringTok{" "}\NormalTok{)}
\end{Highlighting}
\end{Shaded}

\begin{verbatim}
## [1] " "
\end{verbatim}

\begin{Shaded}
\begin{Highlighting}[]
\FunctionTok{print}\NormalTok{(}\StringTok{" "}\NormalTok{)}
\end{Highlighting}
\end{Shaded}

\begin{verbatim}
## [1] " "
\end{verbatim}

\begin{Shaded}
\begin{Highlighting}[]
\FunctionTok{print}\NormalTok{(}\StringTok{" "}\NormalTok{)}
\end{Highlighting}
\end{Shaded}

\begin{verbatim}
## [1] " "
\end{verbatim}

\begin{Shaded}
\begin{Highlighting}[]
\FunctionTok{print}\NormalTok{(}\StringTok{"No number 13"}\NormalTok{)}
\end{Highlighting}
\end{Shaded}

\begin{verbatim}
## [1] "No number 13"
\end{verbatim}

\begin{Shaded}
\begin{Highlighting}[]
\FunctionTok{cor}\NormalTok{(parks\_df[}\SpecialCharTok{{-}}\DecValTok{13}\NormalTok{, ])}
\end{Highlighting}
\end{Shaded}

\begin{verbatim}
##               size  visitors
## size     1.0000000 0.2299564
## visitors 0.2299564 1.0000000
\end{verbatim}

\begin{Shaded}
\begin{Highlighting}[]
\FunctionTok{print}\NormalTok{(}\StringTok{" "}\NormalTok{)}
\end{Highlighting}
\end{Shaded}

\begin{verbatim}
## [1] " "
\end{verbatim}

\begin{Shaded}
\begin{Highlighting}[]
\FunctionTok{print}\NormalTok{(}\StringTok{" "}\NormalTok{)}
\end{Highlighting}
\end{Shaded}

\begin{verbatim}
## [1] " "
\end{verbatim}

\begin{Shaded}
\begin{Highlighting}[]
\FunctionTok{print}\NormalTok{(}\StringTok{" "}\NormalTok{)}
\end{Highlighting}
\end{Shaded}

\begin{verbatim}
## [1] " "
\end{verbatim}

\begin{Shaded}
\begin{Highlighting}[]
\FunctionTok{print}\NormalTok{(}\StringTok{"No 13 or 7"}\NormalTok{)}
\end{Highlighting}
\end{Shaded}

\begin{verbatim}
## [1] "No 13 or 7"
\end{verbatim}

\begin{Shaded}
\begin{Highlighting}[]
\FunctionTok{cor}\NormalTok{(parks\_df[}\FunctionTok{c}\NormalTok{(}\SpecialCharTok{{-}}\DecValTok{7}\NormalTok{, }\SpecialCharTok{{-}}\DecValTok{13}\NormalTok{), ])}
\end{Highlighting}
\end{Shaded}

\begin{verbatim}
##              size visitors
## size     1.000000 0.398539
## visitors 0.398539 1.000000
\end{verbatim}

No the measurement does not have any impact on the correlation. That is
because correlation is a statistical measurement of how two variables
are related ie. if one value changes by one unit, how does the other
change. Therefore, they change the same.

\end{document}
